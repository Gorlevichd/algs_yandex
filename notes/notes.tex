\documentclass[a4paper, 12pt]{article}
\usepackage[a4paper, total={6in, 10in}]{geometry}
\usepackage{cmap}
\usepackage[T2A]{fontenc}
\usepackage[utf8]{inputenc}
\usepackage[russian, english]{babel}
\usepackage{statmath}
\usepackage{amsmath}
\usepackage{amssymb}
\usepackage{color}
\usepackage{graphicx}
\usepackage{dcolumn}
\usepackage[parfill]{parskip}
\usepackage[normalem]{ulem}
\useunder{\uline}{\ul}{}
\usepackage{listings}
\lstset{extendedchars=\true}

\title{Алгоритмы}

\begin{document}

\maketitle

\section{Сложность, тестирования и особые случаи}

\subsection{Сложность алгоритма}

\textbf{Сложность алгоритма} - порядок количества действий, 
которые выполняет алгоритм

\begin{enumerate}
    \item Константы не включаются в O(N)
    \item Асимтотически константы не влияют на скорость
    работы алгоритма при больших параметрах
    \item \textit{Пространственная сложность} - количество использованной
    памяти
    \item Требуется решение с наименьшей асимптотической сложностью
\end{enumerate}

\subsubsection{Задача UTF-8}

\underline{Задача:} 

Дана строка в кодировке UTF-8, найти
самый частый символ.

\underline{Решение $\#1$:}

\begin{enumerate}
    \item Внешний цикл перебирает все позиции
    \item Внутренний цикл для каждой позиции перебирает
    все другие символы и считает совпадения
    \item $O(N^{2})$, память: $O(N)$
\end{enumerate}

\underline{Решение $\#2$}

\begin{enumerate}
    \item Перебираем все символы в строке и
    находим уникальные буквы
    \item Потом перебираем уже уникальные буквы а не все
    \item $O(kN) \sim O(N)$, память: $O(N + K)$
\end{enumerate}

\underline{Решение $\#3$:}

\begin{enumerate}
    \item Создаем словарь, ключ - символ, а значение
    - сколько раз он встречался
    \item Если символ встречается впервые, то инициализируем нулем
    \item Прибавляем к элементу словаря с ключом, совпадающем
    с этим символом, единицы
    \item $O(N + k)$, память: $O(K)$
\end{enumerate}

\subsection{Тестирование}

Что нужно тестировать:

\begin{enumerate}
    \item Тесты из условия (если есть)
    \item Общие случаи
    \item Особые случаи
\end{enumerate}

Советы по составлению тестов:

\begin{enumerate}
    \item Если есть примеры - решить руками и сверить ответ
    \item Проверить последовательность из одного элемента
    и пустую последовательность
    \item "Краевые эффекты" - проверить, что программа работает
    корректно в начале и в конце последовательности
\end{enumerate}
\end{document}